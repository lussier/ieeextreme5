% QUESTION O
%%%%%%%%%%%%%%%%%%%%%%%%%%%%%%

\section{Question O:\\Patron}

\subsection{Description}
Vangelis the bear wants to understand more about music; and giving a 
scientific look into songs, he realized that most songs have patterns 
(a.k.a motives) in them. After making this discovery, Vangelis decided 
to study these reappearing sequences of notes by using his computer and 
to do so he simplified the songs by converting them into 8bit integer 
values that describe only the pitch of a note. Thus, a song can be a 
sequence of the form 2,5,12,145,233,…

\subsection{Task}
Your task is, given one song encoded as above, to identify all the 
patterns that appear at least L times and have size of at least D 
notes and report back the sum of appearances of the patterns found. 
A pattern of size X may contain another pattern of size Y (X>Y). 
Patterns may overlap.

\subsection{Input}
Your program should read the standard input.
The first line of the input contains an integer D representing the 
minimum allowed number of the notes within a pattern (where 1 ≤ D ≤ 500 ).
The second line contains an integer L representing the minimum 
allowed number of appearances of the pattern within the song 
(where 2 ≤ L ≤ 500 ).
The third line contains an integer F representing the number of 
notes within the song ( where 2 ≤ F ≤ 10000 ).
Each of the next F lines contains a single integer in the 
range [0,255], which depicts the discrete note pitch.

\subsection{Output}
Your program should write on the standard output 1 line. The line 
should contain a single integer, in range [2,4875250], denoting 
the sum of appearances of the patterns that appear at least L 
times and contain at least D notes.

Note: There is no new line character at the end of the result.

\subsection{Input}
2\\
2\\
6\\
1\\
2\\
1\\
2\\
1\\
2

\subsection{Output}
11

\subsection{Explanation}
(Pattern) - Occurrences\\
(1 2) - 3\\
(2 1) - 2\\
(1 2 1) - 2\\
(2 1 2) - 2\\
(1 2 1 2) - 2
