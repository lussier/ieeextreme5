% QUESTION I
%%%%%%%%%%%%%%%%%%%%%%%%%%%%%%

\section{Question I:\\Help Bugs Bunny escape Elmer}

\subsection{Description}
A group of Elmer Fudds are chasing Bugs Bunny in a binary tree. In every turn, Bugs
moves and then all the Elmers are moving. Each move is along an edge of the tree. If
any of the Elmers is on the same node as Bugs, before or after Bugs’ move, they win.
The tree's height is H and nodes are numbered from 1 to $ 2^{H-1} $

For example, see tree of height 3.

\includegraphics[width=5cm]{src/qI0.png}

Each move is one of the following four letters\\

U – moving up the tree (for example, from 7 to 3 in the example above)\\
L – moving to the left child (for example, from 1 to 2 in the example above)\\
R – moving to the right child (for example, from 2 to 5 in the example above)\\
S – stay in the same place (for example, from 4 to 4 in the example above)\\

All the Elmers start at the root, while Bugs can start in any node.
Note that Elmer catches Bugs if he is on the same spot before or after the move; or if
they cross each other on their way.

\subsection{Task}
Please write a program that gets the moves of the Bugs and the moves of all the
Elmers and decides when he is caught or whether Bugs has escaped.

\subsection{Input}
The first line contains the number of problems. Each one of the other lines contains
three numbers: the first is N = number of Elmers; the second is M = number of
moves; and the third is P0 = Bugs's initial location. These numbers are followed by M
groups of N+1 letters each which define the moves of Bugs (the first letter) and each
one of the Elmers (the next N letters).

\subsection{Output}
The move number after which Bugs is caught, and the index of the Elmer who caught
it (if several has, give the minimal one), or 0 0 if he survives the search.

Note: There is no new line character at the end of the result.

\subsection{Sample Input}
2\\
4 2 3 LLLRR SLRLR\\
2 6 8 LLR ULS SRS USR USU SUU

\subsection{Sample Output}
2 3\\
0 0
