% QUESTION E
%%%%%%%%%%%%%%%%%%%%%%%%%%%%%%

\section{Question E:\\Kakuro Validation}

\subsection{Description}
Kakuro is a number puzzle also known as “cross-sum”. The Kakuro puzzle 
is a n * n board which has 3 types of cells:

\begin{enumerate}
\item White Cells or Blank Cells in which the solver can put in any 
number between 1 and 9.

\includegraphics{src/qe0.png}

represented as 0 0

\item Clue cells are Grey Cells that carry sums of sequences (row or 
column or both). They are not fillable but provide clues for filling 
values in the row and column sequences that follow.

\includegraphics{src/qe1.png}\includegraphics{src/qe2.png}
\includegraphics{src/qe3.png} represented as -1 17, 14 -1 
and 14, 17 respectively

\item Block Cells are Grey Cells are empty and unused.

\includegraphics{src/qe4.png} represented as -1 -1
\end{enumerate}

The objective of the game is to fill the white cells with 
values that that the rules of the game are met.

The rules of the game are as follows
\begin{enumerate}
\item A digit cannot appear more than once in a sequence (row or column). 
Note that if the row or column has more than one sequence, the digit 
can appear once in each sequence
\item The sum of digits in a row sequence should match the row sum 
in the clue cell that appears ahead of the row sequence.
\item The sum of digits in a column sequence should match the column 
sum in the clue cell that appears ahead of the column sequence.
\end{enumerate}

There are some rules that define if the structure of the Kakuro 
Puzzle is valid

\begin{enumerate}
\item All row sequences of 2 to 9 white cells should be preceded (to the left) 
by a Clue Cell that has a row sum
\item All column sequences of 2 to 9 white cells should be preceded (above) 
by a Clue Cell that has a column sum
\item No clue cells should be present where there is no row sequence 
or column sequence that follows it.
\item Clue Cells should have only a row sum if there is a row sequence 
that follows it but no column sequence below it. Similarly it should only 
have a column sum if there is a column sequence that follows it but no 
row sequence that follows it. It should have a row and column sum if it has 
both row and column sequences that follow.
\end{enumerate}

An example of where both row and column cell are present is shown below

\begin{enumerate}
\item All row and column sums should have values between 3 and 45
\item The White Cells should be contiguous and connected. They should 
not be in the form of multiple unconnected islands. (see sample scenario 4)
\item All White Cells should have both row and column sequence neighbors. 
In other words, no row or column sequence can just have one member cell. 
See example below.
\end{enumerate}

The following diagram depicts some scenarios where the structure is invalid

\begin{enumerate}
\item Row sum instead of column sum
\item Missing row sum
\item Orphan cell that has no column neighbor
\item Row sum with so subsequent row sequence
\end{enumerate}

Additionally, there should be one and only one solution possible 
with the given puzzle clues.

\subsection{Task}
Write a program that when given a potential Kakuro Puzzle, 
identifies if

\begin{enumerate}
\item it is structurally correct
\item it has a possible valid solution
\item it has a unique solution
\end{enumerate}

\subsection{Time Limit}
Maximum three seconds

\subsection{Input}
The first line of the input N is the puzzle board size (4 <= N <= 10)

This is followed by N lines of input (for each row) each containing a 
pair of N numbers that represent the N cells in the row.

-1 -1 represents a grey or block cell

0 0 represents a White or Blank cell that is fillable

-1 R represents a clue cell that carries only a row sequence sum 
(3 <= R <= 45)

C -1 represents a clue cell that carries only a column sequence sum 
(3 <= C <= 45)

C R represents a clue cell that carries both row and column sums

\subsection{Output}
One of the following

If there are structural issues – e.g. missing or extraneous 
row or column sums,

Invalid Puzzle Structure

If it is structurally valid and has a unique solution,

Valid Puzzle

If it is structurally valid but has no valid solution,

No Solution

If it is structurally valid but has multiple valid solutions,

No Unique Solution

\subsection{Sample Input - Scenario 1}
10\\
-1 -1 14 -1 23 -1 -1 -1 6 -1 16 -1 -1 -1 -1 -1 6 -1 6 -1\\
-1 17 0 0 0 0 -1 4 0 0 0 0 -1 -1 4 4 0 0 0 0\\
-1 11 0 0 0 0 13 11 0 0 0 0 29 8 0 0 0 0 0 0\\
-1 -1 -1 16 0 0 0 0 14 17 0 0 0 0 0 0 0 0 -1 -1\\
-1 -1 -1 -1 -1 11 0 0 0 0 0 0 0 0 9 -1 -1 -1 -1 -1\\
-1 -1 -1 -1 -1 7 0 0 0 0 13 8 0 0 0 0 -1 -1 -1 -1\\
-1 -1 -1 -1 16 -1 17 26 0 0 0 0 0 0 0 0 20 -1 -1 -1\\
-1 -1 4 19 0 0 0 0 0 0 0 0 11 10 0 0 0 0 5 -1\\
-1 19 0 0 0 0 0 0 -1 10 0 0 0 0 -1 4 0 0 0 0\\
-1 7 0 0 0 0 -1 -1 -1 4 0 0 0 0 -1 13 0 0 0 0\\

\includegraphics{src/qe5.png}

\subsection{Sample Output - Scenario 1}
Valid Puzzle

\subsection{Explanation - Scenario 1}
There is only one solution as shown below:

\includegraphics{src/qe6.png}

Note that the numbers can appear only once is a given sequence but 
if there are multiple sequences in each row or column, the digit 
can appear once in each of then e.g. 1 and 3 in the second row.

\subsection{Sample Input - Scenario 2}
4
-1 -1 13 -1 11 -1 9 -1
-1 5 0 0 0 0 0 0
-1 6 0 0 0 0 0 0
-1 4 0 0 0 0 -1 -1

\includegraphics{src/qe7.png}

\subsection{Sample Output - Scenario 2}
No Solution

\subsection{Explanation - Scenario 2}
On the second row, no 3 different digits can add to up to 5. 
Hence, no solution possible.

\subsection{Sample Input - Scenario 3}
4\\
-1 -1 13 -1 11 -1 9 -1\\
-1 23 0 0 0 0 0 0\\
-1 6 0 0 0 0 0 0\\
-1 4 0 0 0 0 -1 -1\\

\includegraphics{src/qe8.png}

\subsection{Sample Output - Scenario 3}
No Unique Solution

\subsection{Explanation - Scenario 3}
There are multiple solutions possible. 

Two Examples are shown below:\\

\includegraphics{src/qe9.png} \includegraphics{src/qe10.png}

\subsection{Sample Input - Scenario 4}
6\\
-1 -1 17 -1 3 -1 -1 -1 -1 -1 -1 -1\\
-1 11 0 0 0 0 -1 -1 -1 -1 -1 -1\\
-1 9 0 0 0 0 -1 -1 -1 -1 -1 -1\\
-1 -1 -1 -1 -1 -1 -1 -1 17 -1 3 -1\\
-1 -1 -1 -1 -1 -1 -1 9 0 0 0 0\\
-1 -1 -1 -1 -1 -1 -1 11 0 0 0 0\\

\includegraphics{src/qe11.png}

\subsection{Sample Output - Scenario 4}
Invalid Puzzle Structure

\subsection{Explanation - Scenario 4}
The puzzle (though it has a valid solution) has 2 
disconnected islands of White cells.

\subsection{Sample Input - Scenario 5}
4
-1 -1 13 8 11 -1 9 -1\\
-1 23 0 0 0 0 0 0\\
-1 6 0 0 0 0 -1 -1\\
-1 -1 0 0 0 0 -1 -1\\

\includegraphics{src/qe12.png}

\subsection{Sample Output - Scenario 5}
Invalid Puzzle Structure

\subsection{Explanation - Scenario 5}
The structure is invalid for multiple reasons
\begin{enumerate}
\item The last row has no row sum.
\item The last column has a white cell which is the only cell 
in a column sequence – this is not allowed as all white cells 
should have both row and column neighbors that are white cells
\item The cell with input values 13 8 is incorrect as the row 
sum 8 has no row sequence following it
\end{enumerate}
