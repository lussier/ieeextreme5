% QUESTION A
%%%%%%%%%%%%%%%%%%%%%%%%%%%%%%
\section{Question A\\Computing Pearson’s correlation coefficient} 

\subsection{Description}
Pearson’s correlation coefficient is a statistical measure of correlation
between two data sets or series. The value of the coefficient ranges from
+1 (perfect positive correlation) to -1 (perfect negative correlation). 
Correlations can indicate a predictive relationship that can be useful in 
practical applications. The formula for Pearson’s correlation coefficient 
for two series X and Y is:

$ \rho_{X,Y}={\mathrm{cov}(X,Y) \over \sigma_X \sigma_Y} ={E[(X-\mu_X)(Y-\mu_Y)] \over \sigma_X\sigma_Y} $

where $ \mu_x $ is the population mean of series X (and $ \mu_x $ is the 
same for series Y), $ \mu_x $ is the population standard deviation of series 
X (and $ \mu_x $ the same for series Y), and $ E [] $ is expected value.

\subsection{Task}
Write a program that accepts as input two series of values, X and Y, and 
computes the Pearson correlation coefficient (). You may assume that each 
series has the same number of values.

\subsection{Input}
Series of values X and Y seperated by space.

\subsection{Output}
The program should output the value of $ \rho_{x,y} $ to four decimal 
places. If $ \rho_{x,y} $ cannot be computed, then the output should be 
the string 'invalid input'. 
Note: The input ends with a delimiter. There is no new line character 
at the end of the result.

\subsection{Sample Input}
14 2\\
16 5\\
27 7\\
42 9\\
39 10\\
50 13\\
83 20\\
-\\

\subsection{Sample Output}
The following is the output for the above sample input.\\
0.9856

