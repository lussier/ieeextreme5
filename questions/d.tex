% QUESTION D
%%%%%%%%%%%%%%%%%%%%%%%%%%%%%%

\section{Question D:\\Passing the Number Ball}

\subsection{Description}
The mathematics teacher in a large high school was asked to conduct 
some outdoor games and activities for the entire school and also divide 
the students into 4 different houses. He decided to check their basic 
math skills as well. There was a basket at the center of the playground 
and the place around the basket is divided into 14641 (121 * 121) 
“five feet squares” numbered like a typical X-Y coordinate system. 
He asked all students to distribute themselves equally around the basket 
with 5 feet distance between each other as shown in the yellow cells 
in the figure below. He demonstrated that with 53 students from one class. 
He later planned to do this for the entire school so that he can 
randomly create 4 house teams.

\includegraphics[width=13cm]{src/table_qd.png}

If there are additional students who cannot equally distribute themselves 
around the basket, they would first fill the Eastern edge starting 
from South (as shown in the white cells in the diagram above), and then 
fill the Northern edge starting from East, and then the Western Edge 
starting from North, and finally the Southern edge starting from West.

The students were then asked to list the prime numbers in ascending order 
in the form a spiral, starting with 7 as the number used by the student 
immediately to the East of the basket (refer figure below)

\includegraphics[width=13cm]{src/table_qd2.png}

There were four large similar looking balls given to the students who 
were assigned the largest prime numbers ending in 1, 3,7 and 9 respectively 
(in this case 251, 263, 257 and 239). They had to write their number on 
the ball and then throw it to the next student with the same ending digit 
(in this case 241, 233, 227 and 229) so that they can do the same. This 
process had to repeat till the last student in their team (11, 13, 7, 19)
 who will finally throw the ball into the basket (numbered 5). The team 
 should know who to throw to and who to catch from and any wrong or missed 
 catches meant a penalty. The winning team is the one that has the least 
 penalty and throws the ball into basket in the shortest time.

The sequence of throws for the team with numbers ending in 3 is depicted 
on the figure above.

\subsection{Task}
One of the sharp students decided to write a small program to help their
team. Given their coordinates, it will output the prime number that he 
should write on the ball, the coordinate of the student from who he has 
to catch the ball and the coordinate of the student to whom he had to 
throw the ball.

You task is to help the student write the program.

\subsection{Time Limit}
One second for a maximum of 10000 students and 100 inputs

\subsection{Input}
N – number of students in the high school followed by a list of 
coordinates ‘xi yi’ of students who want to find out who they have 
to catch from and who they have to throw to. (0 < i < 100)

The end of the list is marked by ‘-999 -999’.

\subsection{Output}
For each input coordinate, the coordinates of the thrower and the 
catcher in the form

I should write

I should catch from

I should throw to

Where P(x,y) is the prime number to be written by the student

m, n is the coordinates of the student from whom the catch should 
be taken (if the input is of the first thrower – this should be 
recorded as ‘nobody’ – see example below)

p, q is the coordinates of the student / basket to whom the ball 
should be thrown

if the input location has no student - the the output should just 
have 'No student at this location' for that line 

Note: Each line should end with a new line character.

\subsection{Sample Input}
53

4 7

4 -3

-1 -1

1 1

-3 -1

0 0

-999 -999

\subsection{Sample Output}
No student at this location\\
I should write 239\\
I should catch from nobody\\
I should throw to 2 -3\\
I should write 23\\
I should catch from 2 1\\
I should throw to 0 1\\
I should write 11\\
I should catch from 1 -1\\
I should throw to 0 0\\
I should write 191\\
I should catch from -1 -3\\
I should throw to -3 0\\
No student at this location

