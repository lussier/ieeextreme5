% QUESTION B
%%%%%%%%%%%%%%%%%%%%%%%%%%%%%%
\section{Question B\\Implementing a letter-specific Bloom filter}

\subsection{Description}
A Bloom filter is a space-efficient data structure used for probabilistic 
set membership testing. When testing an object for set membership, a Bloom 
filter may give a false positive (that is, return a “true” for the tested 
object when the object is not mapped into the filter), but never a false 
negative (that is, return a “false” when the tested object is mapped into 
the filter). For this problem you will implement a Bloom filter 
specifically intended for mapping and testing of English words.

\subsection{Task}
Write a program accepts as input a series of mapping words (say, “cat”, 
“dog”, and “bird”) to map or load the Bloom filter as described below. 
Once the Bloom filter is mapped the program should accept words to test 
against the mapped filter. The program should return TRUE or FALSE for 
each entered test word as described below. All inputs are case 
insensitive (that is, “Cat” and “cat” should be considered to be the same).
The Bloom filter is a boolean array indexed from 1 to 26. The filter 
array is initialized to all array values, or positions, set to FALSE. 
Each letter of a mapping word sets the corresponding alphabetical position 
value in the array to TRUE (if it is already TRUE from a previous mapping, 
then it remains TRUE). For example, “Cat” would set array positions 3, 
1, and 20 to TRUE corresponding to position 3 for “C”, 1 for “a”, and 20 
for “t”. When testing for set membership, each position in the filter
array must be TRUE for the positional value of each letter in the test 
word for the program to return TRUE, else FALSE is returned.

\subsection{Input}
The first line contains a series of English words delimited by spaces, 
line breaks, and/or punctuation “.”, “,”, “?”, and “!”.
The second line contains the testing words.

\subsection{Output}
The program should output the number of test words (from the testing file) 
that tested as TRUE in the Bloom filter.

\subsection{Sample Input}
cat dog bird turtle\\
Bottle MOUSE cat doggy Dog

\subsection{Sample Output}
For the above sample input the output should be:\\
3

\pagebreak
