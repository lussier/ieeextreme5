% QUESTION S
%%%%%%%%%%%%%%%%%%%%%%%%%%%%%%

\section{Question S:\\Simple polygons}

\subsection{Description}
Cyclic convex polygons are a simple kind of polygons: all N 
corners lie on a circle and the boundaries do not intersect. 
The surface of such a polygon is area is shown hatched in 
the example sketched below (N=5).

\includegraphics[width=10cm]{src/qs0.png}

The corner positions (xi, yi) can be expressed in terms of 
the angle as shown in the example.

\subsection{Task}
Write a program that reads a set of corner point angles 
from stdin. The radius of the cyclic polygon is 1 (unit circle)
 and the angles are given as multiples of $\pi$ 
 (thus in the range 0..2). Your program calculates the surface area 
 of the specified polygon (N<=200) and outputs it to stdout.

\subsection{Input/Output format}
Each line of the input file contains a corner point angle. The 
output is expected to have the following c-style format: \%.2f.

Keep reading Input until you hit dollar sign \$

Note: There is no new line character at the end of the result.

\subsection{(Isosceles triangle) Input:}
0\\
0.5\\
1\\
\$

\subsection{(Isosceles triangle) Expected output}
1.00\\

